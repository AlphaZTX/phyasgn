\documentclass{phyasgn}\usepackage{nag}
\phyasgn{classname=北京大学物理学院课程作业模板}

%\ctexset{punct=kaiming}
\setCJKmainfont[ItalicFont=FZKTK.TTF,BoldFont=FZXBSK.TTF]{FZSSK.TTF}
\setCJKsansfont[BoldFont=FZHTK.TTF]{FZXH1K.TTF}
\setCJKmonofont[ItalicFont=FZKTK.TTF]{FZFSK.TTF}
\newCJKfontfamily\FZSS{FZSSK.TTF}
\newCJKfontfamily\FZKT{FZKTK.TTF}
\newCJKfontfamily\FZFS{FZFSK.TTF}
\newCJKfontfamily\FZHT{FZHTK.TTF}
\setmainfont{TeX Gyre Termes}
\setsansfont{TeX Gyre Heros}[Scale=MatchLowercase]
\setmonofont{Ubuntu Mono}%[Scale=MatchLowercase]
\newfontfamily\lm{Latin Modern Roman}
\usepackage{unicode-math}
\setmathfont{TeX Gyre Termes Math}

\newcommand\pkg[1]{\textsf{#1}}
\newenvironment{csop}{\vskip\topsep\noindent\hspace{2em}\ttfamily\small\ignorespaces}{\vskip\topsep\par}

\usepackage{booktabs,metalogo,siunitx,marginnote,manfnt,url}
\usepackage[unicode]{hyperref}
\hypersetup{pdfstartview=XYZ,hidelinks,pdfcreator=XeTeX Output,pdfauthor=张庭瑄,
pdftitle=phyasgn文档类}
\usepackage{geometry,fancyhdr}
\geometry{left=3cm,right=3cm,marginparwidth=4em}
\fancyhead[L]
{\begin{tabular}[b]{@{}l@{}}
  \hyperref{https://www.phy.pku.edu.cn/}{}{}{\includegraphics[height=2.12em]{phylogo.pdf}}
\end{tabular}}
\fancyhead[C]
{\begin{tabular}[b]{@{}c@{}}
  \large 
  \pkg{phyasgn}\,文档类
  \\[-2pt]
  {\scriptsize 姓名:~张庭瑄\quad 学号:~{\sffamily xxxxx}11{\sffamily xxx}}
\end{tabular}
}

\usepackage{shortvrb,fancyvrb}
\MakeShortVerb|
\fvset{xleftmargin=2em,fontsize=\small}
\makeatletter
\ifx\l@nohyphenation\undefined
  \newlanguage\l@nohyphenation
\fi
\DeclareRobustCommand\meta[1]{%
  \ensuremath\langle
  \ifmmode \expandafter \nfss@text \fi
  {%
    \rmfamily\itshape
    \edef\meta@hyphen@restore
    {\hyphenchar\the\font\the\hyphenchar\font}%
  \hyphenchar\font\m@ne
  \language\l@nohyphenation
  #1\/%
  \meta@hyphen@restore
  }\ensuremath\rangle
}
\makeatother

\def\phyasgn{\pkg{phyasgn}}
\def\version{0.2 $\upbeta$}

\title{
  {\pkg{phyasgn}\,文档类}\\[-8pt]
  {\normalsize ——北京大学物理学院课程作业模板}\\[-11pt]
  {\normalsize\mdseries Version \version}
}
\author{张庭瑄}
\date{2021 年 9 月 30 日}

\begin{document}
\maketitle

\begin{abstract}
\phyasgn 是基于 \pkg{article} 的文档类, 旨在提供一个风格统一的课程作业模板. 
本模板的中文支持方案是 \pkg{xeCJK} 宏包, 因此需要使用 \XeLaTeX{} 编译.
在这个模板中, 已经定义好了一些数学命令, 方便用户输入.
\end{abstract}



\tableofcontents





\section{模板的安装}
对于 Windows 系统下的 \TeX{} Live, 把模板文件夹 (如果下载的是 |.zip| 压缩包则需要解压) 移动到
\begin{csop}
\meta{安装磁盘}:\textbackslash\meta{安装目录}\textbackslash texlive\textbackslash\meta{版本}\textbackslash texmf-dist\textbackslash tex\textbackslash latex\textbackslash 
\end{csop}
\noindent 目录下或
\begin{csop}
C:\textbackslash Users\textbackslash \meta{用户名}\textbackslash texmf\textbackslash tex\textbackslash
\end{csop}
\noindent 目录下.\footnote{如果不存在此文件夹, 请新建一个.} MiK\TeX{} 与之类似.\footnote{并不建议使用 MiK\TeX{} 发行版, \emph{不要}使用 C\TeX{} 发行版.}

对于 macOS 系统下的 \TeX{} Live, 把模板文件夹移动到
\begin{csop}
usr/local/texlive/\meta{版本}basic/texmf-dist/tex/latex/
\end{csop}
\noindent 目录下. Linux 下的 \TeX{} Live 以及 macOS 下的 Mac\TeX{} 同理.

文件夹包含以下内容:
\begin{itemize}
\item |phyasgn.cls|: 文档类; 
\item |phylogo.pdf|: 物理学院图标;
\item |phyasgn.pdf|: 文档类手册;
\item |phyasgn-example.pdf|: 样例, 可直接查阅.
\end{itemize}


\section{系统要求}
\begin{itemize}
\item \textbf{包含 \XeLaTeX{} 的 \TeX{} 发行版.} 
建议使用 \TeX{} Live, macOS 用户可以考虑 Mac\TeX.
%\item \textbf{\pkg{xeCJK} 宏包.} 
%提供中文支持, 使用 \XeLaTeX{} 排版.
%\item \textbf{\pkg{ctex} 宏集.} 
\item \textbf{TTF 或 OTF 格式的中文字体.}\footnote{在 Windows 10 系统下, 
若需使用自定义中文字体, 需要\emph{为所有用户安装}.}
建议使用支持 GBK 字符集的字体.
%\item \textbf{其他的一些基本的宏包, 如 \pkg{amsmath}, \pkg{amssymb}, \pkg{amsfonts}.} 
%作为基本的数学输入支持.
%\item \textbf{\pkg{txfonts} 宏包.} 
%仅使用 \pkg{txfonts} 中的 |txmia| 字体, %用以实现本模板对数学模式下直立体希腊字母的支持, 
%并不需要完整的宏包.
\item \textbf{以下宏包:}\\
\begin{minipage}{\textwidth}
\centering
\begin{tabular}{lllll}
\toprule
\pkg{amsfonts}&\pkg{amsmath}&\pkg{ctex}&\pkg{enumitem}&\pkg{expl3}\\
\pkg{fancyhdr}&\pkg{footmisc}&\pkg{geometry}&\pkg{graphicx}&\pkg{lastpage}\\
\pkg{multicol}&\pkg{multitoc}&\pkg{tcolorbox}&\pkg{xeCJK}&\pkg{xkeyval}\\
\pkg{xparse}&&&&\\
\bottomrule
\end{tabular}
\end{minipage}
\item \textbf{\textsf{txmia} 字体.} 可以在 \pkg{txfonts} 宏包\footnote{不要使用 
\pkg{txfonts} 宏包, 详见 \ref{不要使用txfonts} 节.} 中找到.
\end{itemize}



\section{选项设置}
\phyasgn 文档类的选项设置在导言区通过下面这种方式实现:
\begin{csop}
\small \textbackslash phyasgn\{ \meta{选项\,1} = \meta{值\,1}, \meta{选项\,2} = \meta{值\,2}, $\cdots$ \}
\end{csop}
%我们给出一个示例:
%\begin{center}\fbox{\begin{minipage}{0.6\textwidth}
%\includegraphics[width=\textwidth]{lipsum-ex.pdf}
%\end{minipage}}
%\label{lipsum-ex}
%\end{center}

\subsection{\texorpdfstring{设置姓名、\,学号}{设置姓名、学号}}
\begin{csop}
stuname = \meta{姓名},
stunum = \meta{学号}
\end{csop}
\noindent 分别用来设置姓名和学号, 显示在页眉中下方. 默认显示
\begin{center}
\fbox{\begin{minipage}{0.8\textwidth}
\hfil{\scriptsize 姓名: \texttt{学生姓名}\quad 学号: {\sffamily xxxxx}11{\sffamily xxx}}\\[-12pt]
\rule{\textwidth}{1pt}\\[-10pt]\mbox{}
\end{minipage}}
\end{center}

\subsection{设置作业次数}
\begin{csop}
setasgnnum = \meta{次数}
\end{csop}
\noindent 用来设置作业次数, 显示在页眉中上方. 默认显示
\begin{center}
\fbox{\begin{minipage}{0.8\textwidth}
\mbox{}\hfil{\large 第 \texttt{NULL} 次作业}\\[-2pt]
\rule{\textwidth}{1pt}\\[-10pt]\mbox{}
\end{minipage}}
\end{center}
如果不需要设置次数, 可以使用
\begin{csop}
asgnnum = false
\end{csop}
\noindent 选项, 使得页眉中上方显示
\begin{center}
\fbox{\begin{minipage}{0.8\textwidth}
\mbox{}\hfil{\large 课~~程~~作~~业}\\[-2pt]
\rule{\textwidth}{1pt}\\[-10pt]\mbox{}
\end{minipage}}
\end{center}


\subsection{设置课程名称}
\begin{csop}
classname = \meta{课程名称}
\end{csop}
\noindent 用来设置课程名称, 显示在页眉右下方. 默认显示
\begin{center}
\fbox{\begin{minipage}{0.8\textwidth}
\mbox{}\hfill{\scriptsize 课程}\\[-12pt]
\rule{\textwidth}{1pt}\\[-10pt]\mbox{}
\end{minipage}}
\end{center}



\section{对于数学模式的一些定义}
\subsection{直立体希腊字母}
本模板中, 使用直立体希腊字母的方式和 \pkg{upgreek} 宏包几乎一样, 都是采用
\texttt{\textbackslash up\meta{希腊字母的英文拼写}} 这种输入方式, 
如输入 ``|$\updelta$|'' 得到 ``$\updelta$''. 
与 \pkg{upgreek} 宏包不同的是, 本模板的直立体希腊字母允许在文本环境中直接输入, 
比如输入 ``|\upalpha 粒子|'' 可得到 ``$\upalpha$ 粒子''. 
但是,\marginpar{\dbend} 如果使用了 \pkg{unicode-math} 宏包, 希腊字母整体的字体会被 
\pkg{unicode-math} 指定, 而且在正文中直接输入直立体希腊字母的命令会报错.

{\footnotesize 本模板没有使用 \pkg{upgreek} 宏包作为数学环境下直立体希腊字母的支持, 
而是使用了 |txmia| 字体作为解决方案. 其中一个原因是, 
如果使用了 \pkg{siunitx} 宏包 (本模板未自动加载 \pkg{siunitx} 宏包).
\pkg{siunitx} 会自动识别希腊字母的解决方案, 如果使用了 \pkg{upgreek} 宏包, 在单位中输入如 
``|\si{k\ohm}|'' 等包含大写希腊字母的单位时, 会将字体切换为 Euler 或 Symbol 字体, 
可能导致字体不统一的现象.}% (与用户习惯有关)

\subsection{\texorpdfstring{常量、\,集合与算符}{常量、集合与算符}}
\begin{itemize}
\item |\e| 得到 $\e$, 只能在数学模式中使用.
\item |\i| 和 |\j|\marginpar{\dbend} 在数学模式中分别得到 $\mathrm{i}$ 和 $\mathrm{j}$, 
在文本环境中则会得到 \i, \j. 注意, 这个定义与 \pkg{hyperref} 宏包冲突, 使用了 
\pkg{hyperref} 宏包后, |\i| 和 |\j| 都不能出现在数学模式中.
\item 数域: 如 |\C| 得到 $\mathbb{C}$.\marginpar{\dbend} 支持的命令包括 
$\mathbb{N}$, $\mathbb{Z}$, $\mathbb{Q}$, $\mathbb{R}$, $\mathbb{C}$,\footnote{若没有载入 
\pkg{unicode-math}, 则使用 \pkg{amsfonts} 的字体; 若载入 
\pkg{unicode-math}, 则使用 \pkg{unicode-math} 指定的字体.  
\pkg{unicode-math} 可以直接用 \texttt{\textbackslash symbb\{\meta{字母}\}}.} 
分别为反斜杠加上对应的大写字母. 注意, 这些命令仍然与 \pkg{hyperref} 宏包冲突.
%\item |\Re| 和 |\Im| 分别得到 $\Re$ 和 $\Im$. 
%$\oldRe$ 和 $\oldIm$ 因为不符合 ISO 标准, 应当放弃使用. 
%但是本模板定义了 |\oldRe| 和 |\oldIm| 命令分别与之对应.
\item |\d|\marginpar{\dbend} 在数学环境中得到 $\dif$, 
在文本环境中得到常规的 \TeX{} 重音符号 |\d| (例如 |\d{o}| 得到 \d{o}). 
本模板还额外封装了 |\dt| 作为 $\dif t$ 的命令, 以便于快捷输入.
|\d| 在数学模式中同样与 \pkg{hyperref} 宏包冲突.

%{\footnotesize |\d| 是本模板最为巧妙的一个设计, 在分数中不会偏移, 
%在函数后的时候会自动在前面留下小空格, 比如 
%$\int_{t_1}^{t_2} L(q,\dot{q},t)\dif t$. 而且不会像 \pkg{physics} 宏包提供的 |\dd| 命令
%一样在后面产生多余的空格.}
\label{hyperref在数学模式中的一些问题}
\end{itemize}
\paragraph{注意事项} 
\pkg{hyperref} 宏包与很多宏包都会产生冲突,\marginpar{\dbend} 
因此不建议在本模板中使用 \pkg{hyperref} 宏包.



\section{其他配置}
\subsection{列表环境}
\phyasgn 对于 |enumerate| 和 |itemize| 环境的格式做出了修改:

|enumerate|:
\begin{center}
\fbox{
\begin{minipage}{0.8\textwidth}
\begin{enumerate}
\item 一级列表
\begin{enumerate}
\item 二级列表
\end{enumerate}
\end{enumerate}
\end{minipage}}
\end{center}

|itemize|:
\begin{center}
\fbox{
\begin{minipage}{0.8\textwidth}
\begin{itemize}
\item[\textbullet] 一级列表
\begin{itemize}
\item[\textendash] 二级列表
\end{itemize}
\end{itemize}
\end{minipage}}
\end{center}
可以直接把 |enumerate| 环境用作题目编号.

\subsection{解与证明类环境}
\phyasgn 设置了不依赖于定理类相关宏包的解与证明类环境, 用于满足解答的需要. 给出例子: 
\begin{flushleft}
\begin{minipage}{\textwidth}
\begin{minipage}{0.48\textwidth}
\begin{Verbatim}
\begin{sol}
foo
\end{sol}
\end{Verbatim}
\end{minipage}
\hfill
\fbox{\begin{minipage}{0.48\textwidth}
\begin{sol}
foo
\end{sol}
\end{minipage}}
\end{minipage}
\end{flushleft}

\begin{flushleft}
\begin{minipage}{\textwidth}
\begin{minipage}{0.48\textwidth}
\begin{Verbatim}
\begin{pf}
bar
\end{pf}
\end{Verbatim}
\end{minipage}
\hfill
\fbox{\begin{minipage}{0.48\textwidth}
\begin{pf}
bar
\end{pf}
\end{minipage}}
\end{minipage}
\end{flushleft}



\section{特殊需求}
\subsection{更换西文字体}
\phyasgn 文档类没有对西文字体做出额外的设置. 一个原因是, \TeX{} 原生的 {\lm Computer Modern} 
字体\footnote{实际上, 在 \XeLaTeX{} 下使用的是 {\lm Latin Modern} 字体.}
足够优秀, 更主要的原因是, 如果使用了不同的字体支持方案, 有可能会产生大大小小的问题, 比如, 
\pkg{unicode-math} 宏包与 \pkg{bm} 宏包冲突, 
前者使用 |\symbfit| 方式实现数学模式下加粗的斜体字母, 与后者的 |\bm| 命令冲突; 
\pkg{mathspec} 宏包又与 \pkg{unicode-math} 宏包冲突,\footnote{详见 
\pkg{unicode-math} 宏包手册. 可在终端 (\textsf{cmd}, \textsf{shell}) 
中执行 ``\texttt{texdoc unicode-math}'' 查阅.} 等等. 

如需自行设置西文正文字体, 推荐使用 \XeLaTeX{} 下的 \pkg{fontspec} 宏包. 不要使用 
\pkg{txfonts}, \pkg{pxfonts} 等宏包.\footnote{\pkg{txfonts} 与 \AmS 系列宏包
的使用顺序不当会造成字体设置失效.}\label{不要使用txfonts}

如需自行设置数学字体, 建议使用 \pkg{unicode-math} 宏包\footnote{不要使用 \pkg{txfonts} 宏包, 
因为有一些数学符号的定义很容易冲突.} 
或 \pkg{newtxmath} 宏包.

\subsection{更换中文字体}
本模板的默认中文字体设置遵循 \pkg{ctex} 宏集的缺省设置. 例如, 对于 Windows 10 用户, 
默认字体设置被 \pkg{ctex} 传递给 |fontset = windowsnew| 或 |fontset = fandol|; 
对于 macOS 用户, 默认字体设置为 |fontset = macnew|; 
对于 Linux 发行版用户, 默认字体设置为 |fontset = fandol|.

如需自行设置中文字体, 可以在导言区设置
\begin{Verbatim}
\phyasgn{ ctexfont = false }
\end{Verbatim}
然后利用 \pkg{xeCJK} 提供的字体设置命令:
\begin{Verbatim}
\setCJKmainfont[ItalicFont=FZKTK.TTF,BoldFont=FZXBSK.TTF]{FZSSK.TTF}
\setCJKsansfont[BoldFont=FZHTK.TTF]{FZXH1K.TTF}
\setCJKmonofont{FZFSK.TTF}
\end{Verbatim}
即可把中文字体设置为方正字库的字体系列.\footnote{{\FZSS 方正书宋}, 
{\FZKT 方正楷体}, {\FZFS 方正仿宋}, {\FZHT 方正黑体}可以免费商用.}


\section{History \& Issues}
\begin{itemize}
\item Version 0.1 $\upbeta$, 2021 年 9 月 12 日. 
\begin{itemize}
\item 基于 \textsf{ctexart} 的 \phyasgn. 仅由作者本人进行了一些测试, 未公开.
\item 部分数学模式的定义在 \pkg{hyperref} 宏包中报错, 是 \pkg{hyperref} 的兼容性引起的.
\item 在 0.1 $\upbeta$ 版本中, 暂时还未对代码环境做出配置.
\item 0.1 $\upbeta$ 版本中还对 $\oldRe$ 和 $\oldIm$ 这两个符号做出了修改, 把 |\Re| 和 |\Im| 
分别定义为 $\operatorname{Re}$ 和 $\operatorname{Im}$, 但是这种修改在使用 \pkg{unicode-math} 
的情况下会失效. 此外, 还对 $\mathrm{i}$ 和 $\mathrm{j}$ 的定义做出了修改, 但是在 \pkg{unicode-math} 
下也会失效.
\end{itemize}
\item Version 0.2 $\upbeta$, 2021 年 9 月 30 日. 
\begin{itemize}
\item 基于 \textsf{article} 重构了 \phyasgn. 是公开发布的第一版.
\item 暂时放弃对 \pkg{unicode-math} 的支持, 后续版本中会把对 \pkg{unicode-math} 的支持加入进来. 
如需更改数学字体, 建议使用 \pkg{newtxmath}.
\item 仍未对代码环境做出配置.
\item 不推荐在本模板中使用 \pkg{physics} 宏包.
\end{itemize}

\end{itemize}








\end{document}